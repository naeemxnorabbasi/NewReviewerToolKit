\documentclass{article}
\title{Peer Review: Script for Slides}
\author{Naeem Abbasi}
\date{Nov 23, 2024}

\begin{document}

\maketitle

\section*{Slide 1: What is Peer Review?}
\begin{itemize}
    \item Introduce the concept of peer review: Experts evaluate work for correctness and quality before publication.
    \item Explain that familiarity with this process benefits future paper submissions.
\end{itemize}

\section*{Slide 2: What is a Double-Blind Review?}
\begin{itemize}
    \item Define a double-blind review: Neither the author nor the reviewer knows each other's identity.
    \item Discuss the importance of fairness and impartiality in the process.
\end{itemize}

\section*{Slide 3: What to Look For When Reviewing a Paper?}
\begin{itemize}
    \item Key elements:
        \begin{itemize}
            \item \textbf{Motivation}: Why the problem matters and the importance of the results.
            \item \textbf{Problem Statement}: The specific problem being addressed.
            \item \textbf{Approach}: How the authors tackle the problem or make progress.
            \item \textbf{Results}: Key findings and their significance.
            \item \textbf{Conclusions}: Broader implications and relevance.
        \end{itemize}
    \item Mention these elements are often in the abstract or introduction.
\end{itemize}

\section*{Slide 4: What Should Be in a Good Review?}
\begin{itemize}
    \item Structure of a helpful review:
        \begin{itemize}
            \item A short summary covering the problem, approach, contributions, and results.
            \item Questions for authors to clarify unclear points.
            \item Editorial comments on grammar, figures, tables, and references.
        \end{itemize}
    \item Emphasize constructive feedback to improve the paper.
\end{itemize}

\section*{Slide 5: Feedback to Authors and Committee}
\begin{itemize}
    \item Two types of feedback:
        \begin{itemize}
            \item \textbf{To Authors}: Helps improve clarity, address weaknesses, and enhance technical quality.
            \item \textbf{To Committee}: Helps decide on acceptance, relevance, originality, accuracy, and clarity.
        \end{itemize}
\end{itemize}

\section*{Slide 6: Submitting the Review}
\begin{itemize}
    \item Explain the use of tools like EasyChair for formal review submissions.
    \item Key elements to include: Decision, confidence level, relevance, originality, clarity, references, and presentation recommendation.
    \item Encourage communication with organizers if unable to complete the review on time.
\end{itemize}

\section*{Slide 7: Thank You}
\begin{itemize}
    \item Thank the audience for participating in this volunteer activity.
    \item Highlight their valuable contributions to the academic community.
    \item Conclude with encouragement to reach out with questions or concerns.
\end{itemize}

\end{document}
