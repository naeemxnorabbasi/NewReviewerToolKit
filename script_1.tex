\documentclass{article}
\title{Peer Review: Script for Slides}
\author{Naeem Abbasi}
\date{Nov 23, 2024}

%\documentclass{article}

%\usepackage{fancyhdr}

%\pagestyle{empty}

\begin{document}

\maketitle

\section*{Slide 1: Welcome and Thank You!?}
Welcome to ISCAS-2025 review guidelines presentation! As a reviewer for ISCAS, your feedback is crucial in helping us select the best papers for the conference.

\section*{Slide 2: Decision Recommendations}
Decision recommendations are based on four main categories:
\begin{itemize}
  \item \textbf{ACCEPT}: Represents excellent technical work \& is well written.
  \item \textbf{MARGINAL ACCEPT}: Needs some changes to enhance its technical or presentation quality.
  \item \textbf{MARGINAL REJECT}: Content, style, examples, description of previous work or results needs major improvement.
\item \textbf{REJECT}: Seriously deficient \& should not be considered for this conference.
\end{itemize}

\section*{Slide 3: Confidence Level}
Confidence level is important in providing constructive feedback. We use a scale from Highest to Lowest, with each level indicating increasing certainty.

\section*{Slide 4: Relevance to the Conference}
Relevance refers to how well the paper addresses the conference topic. We use a scale from Excellent to Very Weak, with each level indicating increasing irrelevance.

\section*{Slide 5: Originality of the work}
Originality refers to the novelty and creativity of the paper. We use a scale from Excellent to Very Weak, with each level indicating increasing originality.

\section*{Slide 6: Presentation Format Recommendation}
Finally, we consider the presentation format. Lecture is best suited for longer papers with in-depth discussions, while poster is suitable for shorter papers or those with visual components.

\section*{Slide 7: Thank You}
Thank you for participating in this scholarly activity. Thank you for volunteering your time and effort.

\end{document}
